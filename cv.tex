\documentclass[letterpaper, 10pt]{article} %norsk

  % Packages
\usepackage[left=2cm, right=2cm, top=3cm]{geometry}
\usepackage{hyperref}
\usepackage{multicol}
\usepackage[usenames,dvipsnames]{color}
\usepackage{graphicx}
\usepackage{url}
\usepackage{longtable}
\usepackage{enumitem}

\usepackage{natbib}
\usepackage{bibentry}
\nobibliography{cv}

%\usepackage{fontawesome} %  \faExternalLink

  % Set up hyperref
\hypersetup{
  colorlinks,
  breaklinks,
  linkcolor=NavyBlue,
  urlcolor=NavyBlue,
  pdftitle={Thies Gehrmann Curriculum Vitae},
  pdfauthor={Thies Gehrmann},
  pdfsubject={Curriculum Vitae},
  pdfcreator={Thies Gehrmann},
  pdfproducer={Thies Gehrmann},
}

  % Geometry 
\textheight 23 cm
\textwidth 16 cm
\oddsidemargin 0 cm
\evensidemargin 0 cm

  % Create a function for quoting
\renewcommand{\quote}[1]{
  ``\emph{#1}''
}

  % Make lists without bullets
\renewenvironment{itemize}{
  \begin{list}{$\cdot$}{
    \setlength{\itemsep}{0pt}
    \setlength{\topsep}{7pt}
    \setlength{\leftmargin}{0pt}
    \setlength{\itemindent}{10pt}
  }
}{
  \end{list}
}   

 % A new section header
\newcommand{\nsection}[1]{
    \section*{\color{MidnightBlue}#1}
  \vspace{-1mm}
}



  % A non italicized item in a section
\newcommand{\oentryni}[2]{
  \noindent
  \begin{tabular}[t]{p{0.2\textwidth}|p{0.8\textwidth}}
    \textbf{\color{BrickRed}#1} & {#2} \\
  \end{tabular}
  \vspace{-2mm}

}

 % An italicized item in a section
\newcommand{\oentry}[3]{
  \noindent
  \begin{tabular}[t]{p{0.2\textwidth}|p{0.8\textwidth}}
    \textbf{\color{BrickRed}#1} & {#2 \textit{#3}} \\
  \end{tabular}
  \vspace{-2mm}
}

\usepackage{paracol}

\newcommand{\entryni}[2]{
	\columnratio{0.2,0.8}
	\setlength{\columnseprule}{0.4pt}
	\setlength{\columnsep}{2em}
	\begin{paracol}{2}
		\begin{leftcolumn}
			\begin{flushright}\noindent\textbf{\color{BrickRed}#1}\end{flushright}
			\vspace{-1em}
		\end{leftcolumn}
		\begin{rightcolumn}
			\begin{flushleft}
				\noindent{#2}
			\end{flushleft}
			\vspace{-1em}
		\end{rightcolumn}
	\end{paracol}
	\vspace{-2mm}
}

% An italicized item in a section
\newcommand{\entry}[3]{
	\columnratio{0.2,0.8}
	\setlength{\columnseprule}{0.4pt}
	\setlength{\columnsep}{2em}
	\begin{paracol}{2}
		\begin{leftcolumn}
			\begin{flushright}\noindent\textbf{\color{BrickRed}#1}\end{flushright}
			\vspace{-1em}
		\end{leftcolumn}
		\begin{rightcolumn}
			\begin{flushleft}
				\noindent{#2} \textit{#3}
			\end{flushleft}
		\vspace{-1em}
		\end{rightcolumn}
	\end{paracol}
		\vspace{-2mm}
}


  % No paragraph indent
\setlength\parindent{10pt}

  % Data definitions
\def\doctitle{\color{BrickRed}Curriculum Vit\ae}
\def\name{\color{MidnightBlue}Thies Gehrmann}
\def\maxdeg{\color{Black}PhD}

%%%%%%%%%%%%%%%%%%%%%%%%%%%%%%%%%%%%%%%%%%%%%%%%%%%%%%%%%%%%%%%%%%%%%%%%%%%%%%%
%%                    END LATEX PREAMBLE                                     %%
%%%%%%%%%%%%%%%%%%%%%%%%%%%%%%%%%%%%%%%%%%%%%%%%%%%%%%%%%%%%%%%%%%%%%%%%%%%%%%%

\begin{document}

\small

\pagestyle{plain}

\label{top}

  % Title
{\huge {\textbf \name} {\footnotesize $\,$ \maxdeg} $\;$ {\textbf \doctitle} }
\vspace{0.1cm}
\hrule
\vspace{0.5cm}

%%%%%%%%%%%%%%%%%%%%%%%%%%%%%%%%%%%%%%%%%%%%%%%%%%%%%%%%%%%%%%%%%%%%%%%%%%%%%%%
%%                    CONTACT & CONTENT INFORMATION                          %%
%%%%%%%%%%%%%%%%%%%%%%%%%%%%%%%%%%%%%%%%%%%%%%%%%%%%%%%%%%%%%%%%%%%%%%%%%%%%%%%

\begin{minipage}[t]{0.50\textwidth}
  (email): \href{mailto:thiesgehrmann@gmail.com}{thiesgehrmann@gmail.com} \\
  (https): \href{https://thiesgehrmann.github.io}{thiesgehrmann.github.io} \\
  (tel NL): +31 (0)640 175 860     \\
  %(tel UK): +44 (0)7517 221 445    \\
  \\
  D.O.B. April $8^{th}$ 1989 \\
  Nationality: German \\%\includegraphics[scale=0.4]{de.eps} German\\
  \hrule
  \vspace{5pt}

\end{minipage}
\hfill
\begin{minipage}[t]{0.30\textwidth}
  %\includegraphics[scale=0.4]{nl.eps}
  \textbf{Private address:} \\
  Morskade 14 \\
  2332 GB Leiden \\
  The Netherlands \\
%  \\
%  \includegraphics[scale=0.4]{nl.eps}
%  \textbf{Work address:} \\
%  EWI HB13.090 \\
%  Mekelweg 4\\
%  2628 CD Delft \\
%  The Netherlands \\
%  \\

  Last updated \today.
\end{minipage}


%%%%%%%%%%%%%%%%%%%%%%%%%%%%%%%%%%%%%%%%%%%%%%%%%%%%%%%%%%%%%%%%%%%%%%%%%%%%%%%
%%                      PERSONAL PROFILE                                     %%
%%%%%%%%%%%%%%%%%%%%%%%%%%%%%%%%%%%%%%%%%%%%%%%%%%%%%%%%%%%%%%%%%%%%%%%%%%%%%%%

\nsection{Personal profile}
  \label{sec:persprof}

  Currently, I am a Postdoc in the department of Molecular Epidemiology at the Leiden University Medical Centre, working on translational research of healthy ageing.
  %Having been raised in Norway, among people from all over the world, I am accustomed to an international, multicultural environment.
  Bioinformatics is a constantly developing field, with continuously evolving measurements that presents a splendid opportunity to develop and apply novel methods.
  It also requires the practitioner to develop multidisciplinary skills from the statistical, computational and biological domains.
  I am excited to be part of this field, where my knowledge and competences are constantly challenged, humbled and developed.
  %I am looking for a position where challenging problems, involving many different skills and team work, allow me to contribute positively to society, and further develop my skills.
  
  
%%%%%%%%%%%%%%%%%%%%%%%%%%%%%%%%%%%%%%%%%%%%%%%%%%%%%%%%%%%%%%%%%%%%%%%%%%%%%%%
%%                         EMPLOYMENT                                        %%
%%%%%%%%%%%%%%%%%%%%%%%%%%%%%%%%%%%%%%%%%%%%%%%%%%%%%%%%%%%%%%%%%%%%%%%%%%%%%%%

\nsection{Employment}
	\label{sec:employment}
	
	\entry{Jan. 2018 - Present}
	      {Postdoc at Department of Molecular Epidemiology, Leiden University Medical Centre \newline}
	      {Bioinformatician on healthy ageing, together with Max Plank Institute of Biological Ageing. Within the ERC funded "Geroprotect" project.}
	      
	\entry{Nov. 2016 - 2017}
	      {Postdoc at KNAW/Westerdijk Institute of Fungal Biodiversity \newline}
	      {Embedded bioinformatician. }
	      
	\entry{Nov. 2012 - 2016}
	      {PhD Candidate \newline}
	      {Conducted research and performed teaching. Within the STW funded "Push the white button" project.}

%  \entry{January 2012 - \newline June 2012 }
%        {Teaching Assistant - Course Computational Molecular Biology \newline}
%        {I assisted with the course \textit{Computational Molecular Biology}, part of the Master's degree in Bioinformatics at Leiden University.
%         This involved assisting with the material, creating and grading assignments, and collecting datasets for course use.
%        }


%%%%%%%%%%%%%%%%%%%%%%%%%%%%%%%%%%%%%%%%%%%%%%%%%%%%%%%%%%%%%%%%%%%%%%%%%%%%%%%
%%                        EDUCATION                                          %%
%%%%%%%%%%%%%%%%%%%%%%%%%%%%%%%%%%%%%%%%%%%%%%%%%%%%%%%%%%%%%%%%%%%%%%%%%%%%%%%

\nsection{Education}
  \label{sec:education}

  \entry{2012-2018}
        {Pattern Recognition and Bioinformatics group, TU Delft, The Netherlands \newline}
        {PhD in Bioinformatics. \newline
         \textit{Bioinformatic Analysis of Genetic and Transcriptomic Variation in Fungi}.}

  \entry{2010-2012}
        {Leiden University, Leiden, The Netherlands (In cooperation with TU Delft) \newline}
        {MSc Computer Science Track Bioinformatics. \newline
         Thesis on protein function prediction using Conditional Random Fields.}

  \entry{2008-2010}
        {Heriot Watt University, Edinburgh, Scotland \newline}
        {BSc (Ord) Computer Science \newline
         Graduated with distinction. 
         }

  \entry{2007-2008}
        {Napier University, Edinburgh, Scotland \newline}
        {Certificate of Higher Education \newline
         Completed year 1 of Computer Studies, then transferred to Heriot Watt University.
         }

%  \entry{2000-2007}
%        {International School of Stavanger, Stavanger, Norway \newline}
%        {Completed the IB and IGCSE diplomas.}





%%%%%%%%%%%%%%%%%%%%%%%%%%%%%%%%%%%%%%%%%%%%%%%%%%%%%%%%%%%%%%%%%%%%%%%%%%%%%%%
%%                           SKILLS                                          %%
%%%%%%%%%%%%%%%%%%%%%%%%%%%%%%%%%%%%%%%%%%%%%%%%%%%%%%%%%%%%%%%%%%%%%%%%%%%%%%%

\nsection{Skills}
  \label{sec:skills}

  \entryni{Analytical}
          {Machine learning and statistics; \textit{"Data Science"}. \newline
          	Analysis of Sanger, NGS and long-read sequencing data (genomic, transcriptomic, epigenomic). \newline
          	Genetic/Genomic analysis (Genome assembly, Linkage analysis, *-WAS \& *-QTL, Comparative genomics, Phylogenetic analysis, Variant calling) \newline
          	Transcriptomic analysis (Differential expression, Alternative splicing, Variant calling, eQTL analysis, Timeseries analysis). \newline
          	Epigenetic analysis (Methylation), and metabolomic data.
          }

  \entryni{Technical}
           {Python, Scala, C/C++, Java, Shell, Matlab, R\newline
           SQL, XML (and related technologies), \LaTeX. \newline
           Vector image editing (Inkscape, Affinity Designer). \newline
           Pipeline development and management (Snakemake, Workflow Description Language, Anaconda, Jupyter Notebook). \newline
           Experience with schedulers (SLURM, PDB) and version control software, (Git, SVN).\newline
           Experienced with Unix environments (Linux, MacOS).
           }
  \entryni{Languages}
          {Native speaker, English Norwegian and German.\newline
			High level Dutch. Beginners French.
          }

%%%%%%%%%%%%%%%%%%%%%%%%%%%%%%%%%%%%%%%%%%%%%%%%%%%%%%%%%%%%%%%%%%%%%%%%%%%%%%%
%%                         ACTIVITIES                                        %%
%%%%%%%%%%%%%%%%%%%%%%%%%%%%%%%%%%%%%%%%%%%%%%%%%%%%%%%%%%%%%%%%%%%%%%%%%%%%%%%

%\nsection{Activities}
%  \label{sec:activities}
%
%  \entry{Spring-Autumn 2012}
%        {Master's Thesis on protein function prediction using Conditional Markov Fields\newline}
%        {Changing a previous Markov Random Field model to a CRF model to demonstrate that CRFs can be used to combine several sources of information to predict protein function.
%         \newline}
%
%  \entry{Autumn-Winter 2011}
%        {Literature research project as part of Master's degree \newline}
%        {As a study of contextually aware machine learning methods which can utilize heterogeneous data, Conditional Random Fields were studied.
%         Their use in Bioinformatics was examined in several applications, including gene finding, named entity recognition and protein-protein interaction site prediction.
%        }
%
%  \entry{Spring 2010}
%        {Software development project at Heriot Watt University \newline}
%        {Part of a continuous assessment at Heriot Watt University.
%         The involved writing primitive CGI libraries in C;
%         Thies was responsible for the development and served as project manager.
%         }
%
%  \entry{Summer 2008}
%        {Internship at Napier University \newline}
%        {An attempt was made to develop code for the Silabs CP220x embedded ethernet controller which compiled with the open source SDCC (Small Device C Compiler) to replace proprietary web server code written for proprietary Keil compilers and libraries.
%        }




%%%%%%%%%%%%%%%%%%%%%%%%%%%%%%%%%%%%%%%%%%%%%%%%%%%%%%%%%%%%%%%%%%%%%%%%%%%%%%%
%%                       APPLICATION SHOWCASE                                %%
%%%%%%%%%%%%%%%%%%%%%%%%%%%%%%%%%%%%%%%%%%%%%%%%%%%%%%%%%%%%%%%%%%%%%%%%%%%%%%%

%\nsection{Application Showcase}
%  \label{sec:showcase}
%
%  \entry{DT-cut}
%        {A tool and generalized toolbox for discovering significant clusters in any data.\newline}
%        {\url{https://github.com/thiesgehrmann/DTcut}
%        \newline}
%
%  \entry{ARA}
%        {A toolkit for marker and SNP analyis and quantification. Developed with Thomas Abeel and Arlin Keo.\newline}
%        {\url{https://github.com/AbeelLab/ara}
%        \newline}
%
%  \entry{Proteny}
%        {A toolbox developed with Python to perform a Synteny analysis over two genomes.\newline}
%        {\url{https://github.com/thiesgehrmann/proteny}
%        \newline}
%
%  \entry{RNA-Seq Pipeline}
%        {A pipeline developed with Python and shell tools to perform analysis on RNA-Seq data.\newline}
%        {\url{https://github.com/thiesgehrmann/delftrnaseq}
%        \newline}
%
%  \entry{CRFunc}
%        {A tool developed to integrate multiple sources of data for protein function prediction using conditional random fields.\newline}
%        {\url{https://github.com/thiesgehrmann/CRFunc}}

%%%%%%%%%%%%%%%%%%%%%%%%%%%%%%%%%%%%%%%%%%%%%%%%%%%%%%%%%%%%%%%%%%%%%%%%%%%%%%%
%%                         REFEREES                                          %%
%%%%%%%%%%%%%%%%%%%%%%%%%%%%%%%%%%%%%%%%%%%%%%%%%%%%%%%%%%%%%%%%%%%%%%%%%%%%%%%


\nsection{Referees}
  \label{sec:referees}
%
  \entry{Upon request}{Contact Thies for referee contact details.}{}
%
%  \entry{Thomas Abeel}
%        {Assistant professor in the Delft Bioinformatics Lab \newline}
%        {\href{mailto:T.Abeel@tudelft.nl}{T.Abeel@tudelft.nl} \newline
%         Doctoral co-promotor and daily supervisor.
%         \newline}

%  \entry{Marcel Reinders}
%        {Group leader of the Delft Bioinformatics Lab \newline}
%        {\href{mailto:M.J.T.Reinders@tudelft.nl}{M.J.T.Reinders@tudelft.nl} \newline
%          Doctoral promoter and supervisor.
%         \newline}

%  \entry{Anton Sonnenberg}
%        {Leader of Mushroom Research Group, Wageningen University\newline}
%        {\href{mailto:anton.sonnenberg@wur.nl}{anton.sonnenberg@wur.nl} \newline
%         Collaborator and STW Project user committee member.
%        }
%
%  \entry{Han W\"{o}sten}
%        {Professor in Microbiology at Utrecht University\newline}
%        {\href{mailto:H.A.B.Wosten@uu.nl}{H.A.B.Wosten@uu.nl} \newline
%         Collaborator
%         \newline}

%  \entry{Dick de Ridder}
%        {Professor in Bioinformatics in the WU Plant Sciences group \newline}
%        {\href{mailto:dick.deridder@wur.nl}{dick.deridder@wur.nl}  \url{http://www.wageningenur.nl/en/Persons/prof.dr.ir.-D-Dick-de-Ridder.htm}\newline
%         Master's thesis supervisor.
%         \newline}

%  \entry{Marco Loog}
%        {Assistant professor at the TU Delft Computer Vision lab \newline}
%        {\href{mailto:m.loog@tudelft.nl}{m.loog@tudelft.nl} \url{http://visionlab.tudelft.nl/users/marco-loog}\newline
%         Master's Thesis supervisor.
%         \newline}

%  \entry{Erwin Bakker}
%        {Assistant professor at the Leiden Institute of Advanced Computer Science \newline}
%        {\href{mailto:erwin@liacs.nl}{erwin@liacs.nl}  \url{http://www.liacs.nl/~erwin/}\newline
%         Very familiar with Thies' academic performance.
%         \newline}

%  \entry{Murdoch Gabbay}
%        {Lecturer at the department of Mathematical and Computer Sciences at HWU \newline}
%        {\href{mailto:gabbay@macs.hw.ac.uk}{gabbay@macs.hw.ac.uk}  \url{http://www.macs.hw.ac.uk/~gabbay}\newline
%         He offered considerable advice and inspiration.
%        }

%%%%%%%%%%%%%%%%%%%%%%%%%%%%%%%%%%%%%%%%%%%%%%%%%%%%%%%%%%%%%%%%%%%%%%%%%%%%%%%
%%                       EDUCATIONAL ACTIVITIES                              %%
%%%%%%%%%%%%%%%%%%%%%%%%%%%%%%%%%%%%%%%%%%%%%%%%%%%%%%%%%%%%%%%%%%%%%%%%%%%%%%%

\nsection{Teaching}
  \label{sec:teaching}

  \entry{MSc Courses}{\textit{Frontiers of Science}}
		{ \newline {\normalfont Teaching assistant. Fall 2019 at Leiden University Medical Center. } }

  \entry{}{\textit{Functional Genomics and Systems Biology}}
        { \newline {\normalfont Lecturer and teaching assistant. Fall 2014, 2015, and 2016 at TU Delft } }
        
  \entry{}{\textit{Computational Molecular Biology}}
		{ \newline {\normalfont Teaching assistant. Spring 2012 at Leiden University }  }

  \entry{BSc Courses}{\textit{Biomedical Informatics}}
		{ \newline { \normalfont Lecturer. Fall 2019 at Leiden University Medical Center.} }
		
  \entry{}{\textit{Clinical Research in Practice}}
		{ \newline { \normalfont Lecturer. Fall 2019 at Leiden University Medical Center.} }
        
  \entry{}{\textit{Clinical Biotechnology: Bioinformatics}}
	{ \newline { \normalfont Teaching assistant. Fall 2018, 2019 at TU Delft.} }

  \entry{}{\textit{Molecular Data Science}}
	{ \newline { \normalfont Teaching assistant. Spring 2018 at Leiden University Medical Center } }
      
  \entry{}{\textit{Life Science and Technology Bioinformatics}}
		{ \newline { \normalfont Teaching assistant. Spring 2016 at TU Delft.} } 
      
  \entry{}{\textit{Genome Scale Data Analysis}}
		{ \newline { \normalfont Lecturer and teaching assistant. Fall 2014 at TU Delft } }
		
  \entry{External Courses}{\textit{Quantitative biology summer school}}
        { \newline { \normalfont Lectured on computational aspects of synteny and alternative splicing in fungi. Summer 2015 at Utrecht University.}}

\nsection{Student Supervision}
  \label{sec:supervision}
  \entry{Myrthe de Haan}{B.Sc student 2019-2020\newline}
           {Using the drug-protein interactome to predict intervention-mimicing drugs.}
           
  \entry{Jet Beekwilder}{B.Sc Student 2017-2018\newline}
        {Comparison of DNA extraction methods from fungi for Nanopore sequencing.}
        
  \entry{Andr\'{e} Vollering}{M.Sc Student 2014-2015 \newline}
        {Heterosis: Finding Associated Genomic Regions. (Under embargo) }
        
  \entry{Valerie Pourquie}{B.Sc Student 2015 \newline}
        {\href{https://doi.org/10.1093/gbe/evy121}{Conservation of polarization proteins in yeast and fungi.} }
        
  \entry{Dimitris Palachanis}{M.Sc Student 2013-2014 \newline}
        {\href{http://resolver.tudelft.nl/uuid:8233ed00-a892-4b18-beda-057e539a7d73}{Using the Multiple Instance Learning framework to address differential regulation.} }

%%%%%%%%%%%%%%%%%%%%%%%%%%%%%%%%%%%%%%%%%%%%%%%%%%%%%%%%%%%%%%%%%%%%%%%%%%%%%%%
%%                       OTHER SKILLS/HOBBIES                                %%
%%%%%%%%%%%%%%%%%%%%%%%%%%%%%%%%%%%%%%%%%%%%%%%%%%%%%%%%%%%%%%%%%%%%%%%%%%%%%%%

\nsection{Other activities/skills/information}
  \label{sec:oinfo}

  \bibliographystyle{plainnat}
  \entryni{Awards}
          {Nature Ageing, Rejuvination and Health Conference 2019, 'Best Poster' award.\newline
            ECCB 2016 Student Symposium, 'Best Oral Presentation' award.\newline
          	Benelux Bioinformatics Conference 2015 Student Symposium, 'Best Oral presentation'.\newline
           TU Delft Graduate School, 'Best poster presentation' award 2013.\newline
           Scott Logic Computer Science Prize 2010. } 
  \entryni{Reviewing}
          {\href{https://publons.com/researcher/1615872/thies/peer-review/}{Publons profile}\newline
          	Nature Communications (2019) \newline
          	Bioinformatics (2019) \newline
          	Nature Scientific Reports (2019)\newline
          	FEMS Yeast Research (2018)\newline
          	Nucleic Acid Research (2016)
          }
  \entryni{Volunteering}
          {International Student Network (Leiden) 2013-2019. Set construction and management.\newline
           Dutch Red Cross (Den Haag) 2014-2015. First Aid (EHBO) at events and refugee centers.}
       
	\entryni{Hackathons}{Hackathon for Good 2019, The Hague: \href{https://www.hackathonforgood.org/translators-without-borders}{Translators Without Borders challenge}\newline
	Hackathon for Good 2018, The Hague: \href{https://www.hackathonforgood.org/v1-asser-institute-2018}{Asser Lang-grabbing challenge}}
%  \entryni{Professional Courses followed}
%          {2014 - NBIC: Comparative Genomics.\newline
%           2013 - NBIC: Advanced NGS - RNA-Seq.\newline
%           2014 - CSHL: Statistical Methods in Functional Genomics\newline
%          }
%  \entryni{Transferable skills}
%           2013 - TUD: Art of presenting science.\newline
%           2013 - TUD: Teaching and active learning.\newline
%           2013 - TUD: Coaching individual students.\newline
%           CSHL: Statistical Methods in Functional Genomics.\newline
%          }
%  \entryni{Special Interests}
 %         {Thies regularly enjoys Catamaran/Dinghy sailing, Running, Climbing/Bouldering, and Swimming. Holidays Skiing and Hiking. Arboriculture and arborist climbing techniques. From seed-polyculture vegetable gardening.
%          }

%%%%%%%%%%%%%%%%%%%%%%%%%%%%%%%%%%%%%%%%%%%%%%%%%%%%%%%%%%%%%%%%%%%%%%%%%%%%%%%
%%                         PUBLICATIONS                                      %%
%%%%%%%%%%%%%%%%%%%%%%%%%%%%%%%%%%%%%%%%%%%%%%%%%%%%%%%%%%%%%%%%%%%%%%%%%%%%%%%

\newpage

\nsection{Publications}
\label{sec:publications}
\entryni{}{\href{https://scholar.google.nl/citations?user=-QVv0WkAAAAJ}{Google Scholar profile}}

\entryni{Journal \newline publications}
{
	\begin{enumerate}[label={[\arabic*]},nosep]
		\item \bibentry{guarnaccia2019phyllosticta} \newline
		
		\item \bibentry{vu2019large} \newline
		
		\item \bibentry{diepeveen2018patterns} \newline
		
		\item \bibentry{gehrmann2018nucleus} \newline
		
		\item \bibentry{boonekamp2018genetic} \newline
		
		\item \bibentry{manson2017mycobacterium} \newline
		
		\item \bibentry{pelkmans2017transcription} \newline
		
		\item \bibentry{gehrmann2017variants}
		
		\item \bibentry{pelkmans2016transcriptional} \newline
		
		\item \bibentry{gehrmann2015proteny} \newline
	\end{enumerate}
}

\entryni{Conference \newline publications}
{
		\vspace*{-\baselineskip}
	\begin{enumerate}[label={[\arabic*]},nosep]
		\addtocounter{enumi}{10}
		\item \bibentry{gehrmann2013conditional}
	\end{enumerate}
}

\entryni{Under Review}
{
		\vspace*{-\baselineskip}
	\begin{enumerate}[label={[\arabic*]},nosep]
		\addtocounter{enumi}{11}
		\item {Design and experimental evaluation
			of a minimal, innocuous watermarking
			strategy to distinguish near-identical
			DNA and RNA sequences. (\textit{Under review in ACS Synthetic Biology}) }

	\end{enumerate}
}

\entryni{In Progress}
{
		\vspace*{-\baselineskip}
	\begin{enumerate}[label={[\arabic*]},nosep]
		\addtocounter{enumi}{12}
		\item {A combined lifestyle intervention induces a sensitization of the blood transcriptomic response to a nutrient challenge (\textit{In progress}) } \newline
		\item {Characterization of genetic variants in RAS/MEK/ERK Pathway in exceptionally long-lived Dutch families (\textit{In progress}) } \newline
		\item {Characterization of genetic variants in Linkage regions identified in exceptionally long-lived Dutch families (\textit{In progress}) } \newline
		%\item {Archana Tare, Seungjin Ryu, Cristina Giuliani, Thies Gehrmann, Gil Atzman, Nir Barzilai, Daniela Mari, Giuseppe Passarino, Claudio Franceschi, Eline P Slagboom, Yousin Suh. Genetic landscape of GDF11 and MSTN in human longevity revealed by high-throughput sequencing in a multi-cohort study  (\textit{In progress}) } \newline
		\item {Antonis Somarakis, Manolis Fragkiadakis, Marios Kefalas, Stelios Paraschiakos, Michaelis Vrachasotakis, Thies Gehrmann. On the global prediction of land grabbing risk.  (\textit{Under review in Land}) } \newline
		\item {Annick Lang, Thies Gehrmann, Nils Cronberg. Genetic diversity in bryophyte population with facultative nannandry (\textit{In progress}) } \newline
		%\item {Lizel Mostert, Thies Gehrmann, Duong Vu, Vincent Robert. Genomic analysis of the Grape vine pathogen \textit{Phaeoacremonium} in South African vineyards.  (\textit{In progress}) }

	\end{enumerate}
}

\nsection{Talks}
\entryni{Conference talks}{}
\entry{2019}{DUSRA 2019: Presenter\newline}{A combined lifestyle intervention induces a sex-specific sensitization of the blood transcriptomic response to a nutrient challenge}
\entry{}{LANDac 2019: Panel organizer and presenter\newline}{Uniting Global and Hyper-Local Data for Land: The use of global and local data for land grabbing risk prediction}
\entry{}{European Dialogue on Internet Governance, EuroDIG 2019: Panel organizer and presenter \newline}{Digital cooperation in action - A collaborative case study: Predicting Land Grabbing}
\entry{}{BioSB 2019: Presenter\newline}{Nutrient challenge exposes transcriptomic shift by lifestyle intervention in healthy participants}
\entry{2016}{ECCB Student Symposium 2016: Presenter\newline}{Karyollele specific expression in Agaricus bisporus}
\entry{2015}{BBC Student Symposium 2015: Presenter\newline}{Alternative Splicing in mushrooms from RNA-Seq}
\entry{2013}{Pattern Recognition in Bioinformatics 2013: Presenter\newline}{Conditional Random Fields for protein function prediction}

\entryni{Invited talks}{}
\entry{2014}{Heriot Watt University, Computer Science Seminar Series \newline}{Detecting and visualizing statistically significant clusters of conserved genes of diverged genomes.}

\nsection{Conference Organization}

\entry{2019}{LANDac 2019: Panel organizer\newline}{Uniting Global and Hyper-Local Data for Land}
\entry{}{European Dialogue on Internet Governance, EuroDIG 2019: Panel organizer\newline}{Digital cooperation in action - A collaborative case study}

%\nsection{Posters}
%\entry{2019}{}{}

\nobibliography{cv}
\end{document}

