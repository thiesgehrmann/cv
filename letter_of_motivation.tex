\documentclass[letterpaper, 11pt]{article} %norsk

  % Packages
\usepackage[left=2cm, right=2cm, top=3cm]{geometry}
\usepackage{hyperref}
\usepackage{multicol}
\usepackage[usenames,dvipsnames]{color}
\usepackage[dvips]{graphicx}
\usepackage{url}
\usepackage{longtable}
\usepackage{enumitem}
\usepackage{xspace}
\usepackage{xstring}

\usepackage{natbib}
\usepackage{bibentry}
\nobibliography{cv}


  % Set up hyperref
\hypersetup{
  colorlinks,
  breaklinks,
  linkcolor=NavyBlue,
  urlcolor=NavyBlue,
  pdftitle={Thies Gehrmann Letter of Motivation},
  pdfauthor={Thies Gehrmann},
  pdfsubject={Letter of Motivation},
  pdfcreator={Thies Gehrmann},
  pdfproducer={Thies Gehrmann},
}

  % Geometry 
\textheight 23 cm
\textwidth 16 cm
\oddsidemargin 0 cm
\evensidemargin 0 cm

  % Create a function for quoting
\renewcommand{\quote}[1]{
  ``\emph{#1}''
}

  % Make lists without bullets
\renewenvironment{itemize}{
  \begin{list}{$\cdot$}{
    \setlength{\itemsep}{0pt}
    \setlength{\topsep}{7pt}
    \setlength{\leftmargin}{0pt}
    \setlength{\itemindent}{10pt}
  }
}{
  \end{list}
}

 % A new section header
\newcommand{\nsection}[1]{
    \section*{\color{MidnightBlue}#1}
  \vspace{-1mm}
}

  % A non italicized item in a section
\newcommand{\entryni}[2]{
  \noindent
  \begin{tabular}[t]{p{0.2\textwidth}|p{0.8\textwidth}}
    \textbf{\color{BrickRed}#1} & {#2} \\
  \end{tabular}
  \vspace{-2mm}

}

 % An italicized item in a section
\newcommand{\entry}[3]{
  \noindent
  \begin{tabular}[t]{p{0.2\textwidth}|p{0.8\textwidth}}
    \textbf{\color{BrickRed}#1} & {#2 \textit{#3}} \\
  \end{tabular}
  \vspace{-2mm}
}

\newcommand{\fl}[1]{
  {\color{MidnightBlue}\StrLeft{#1}{1}\xspace}\StrGobbleLeft{#1}{1}\xspace
}

  % No paragraph indent
\setlength\parindent{10pt}

  % Data definitions
\def\doctitle{\color{BrickRed}Letter of Motivation}
\def\name{\color{MidnightBlue}Thies Gehrmann}
\def\maxdeg{\color{Black}M.Sc}

%%%%%%%%%%%%%%%%%%%%%%%%%%%%%%%%%%%%%%%%%%%%%%%%%%%%%%%%%%%%%%%%%%%%%%%%%%%%%%%
%%                    END LATEX PREAMBLE                                     %%
%%%%%%%%%%%%%%%%%%%%%%%%%%%%%%%%%%%%%%%%%%%%%%%%%%%%%%%%%%%%%%%%%%%%%%%%%%%%%%%

\begin{document}

\small

\pagestyle{plain}

\label{top}

  % Title
{\huge {\textbf \name} {\footnotesize $\,$ \maxdeg} $\;$ {\textbf \doctitle} }
\vspace{0.1cm}
\hrule
\vspace{0.5cm}

%%%%%%%%%%%%%%%%%%%%%%%%%%%%%%%%%%%%%%%%%%%%%%%%%%%%%%%%%%%%%%%%%%%%%%%%%%%%%%%
%%                    CONTACT & CONTENT INFORMATION                          %%
%%%%%%%%%%%%%%%%%%%%%%%%%%%%%%%%%%%%%%%%%%%%%%%%%%%%%%%%%%%%%%%%%%%%%%%%%%%%%%%

\noindent\makebox[\textwidth][c]{%
\begin{minipage}[t]{0.9\textwidth}
  \setlength{\parindent}{2em}
  \setlength{\parskip}{1em}

Dear Dr. Vincent Robert,

\fl{I} am applying for the postdoctoral position at CBS/KNAW (CBS-2016-01) that was forwarded to me by Prof. Robin Ohm.
Currently, I am a PhD student under Marcel Reinders and Thomas Abeel at the Delft University of Technology, studying gene regulation throughout mushroom fructification.

\fl{I} have become more and more fascinated by fungi and their incredible, untapped potential to change the world.
My experience with fungi stems mostly from tight collaboration with the group of Prof. Han W\"osten at Utrecht Universit.
This has has been primarily in mushrooms and the process of mushroom fructification.
Through this collaboration, I have become acutely aware how fungi can be used to sustainably replace otherwise harmful solutions in medical, industrial and ecological settings.
Therefore, this position is a great opportunity for me to continue studying fungi to promote their exploitation as bioreactors and to further the understanding of their ecological impact.

\fl{During} my PhD I have performed bioinformatics analyses in two mushrooms, \textit{Schizophyllum commune} and \textit{Agaricus bisporus}.
I developed an algorithm to study the distant synteny between these two mushrooms, discovering chromosomal rearrangements and gene duplications that would not have been found with other algorithms.
I have also performed RNA-Seq analysis in both, aiding in the discovery of additional transcription factors involved in fructification.
Furthermore, I have studied aspects of regulation in these mushrooms that were never before explored: Alternative splicing in \textit{S. commune} and homokaryon specific expression in \textit{A. bisporus}.
The latter work is done in collaboration with Robin Ohm at the University of Utrecht, and Anton Sonnenberg at Wageningen University.
Through these projects, I have come to learn that the mechanisms of regulation are poorly understood or unknown in fungi, especially in mushrooms.

\fl{Recently}, I have also become involved in \textit{Mycobacterium tuberculosis} research.
Through phylogenetic analysis of many TB strains we can identify SNPs responsible for drug resistance, and the evolutionary order of these immunities.
Further, we can discover markers to identify mixed infections in infected patients.

\fl{As} I understand the proposed project described in the announcement, it will address exactly the issues and problems I have come across in my PhD.
Through comparative analysis of thousands of strains, the project will further the understanding of gene and metabolic regulation, drug resistance and the evolutionary relationships and events between strains.
Performing these analyses will contribute significantly to the implementation of fungi in industrial settings.
I believe that this position fits my background of bioinformatics analysis, and also my personal career goals.

\fl{Thank} you for considering my application, and I hope to hear from you.

Kind regards,

-Thies Gehrmann

\end{minipage}
}


\thispagestyle{empty}
\end{document}

