\documentclass[letterpaper, 11pt]{article} %norsk

  % Packages
\usepackage[left=2cm, right=2cm, top=3cm]{geometry}
\usepackage{hyperref}
\usepackage{multicol}
\usepackage[usenames,dvipsnames]{color}
\usepackage[dvips]{graphicx}
\usepackage{url}
\usepackage{longtable}
\usepackage{enumitem}
\usepackage{xspace}
\usepackage{xstring}

\usepackage{natbib}
\usepackage{bibentry}
\nobibliography{cv}


  % Set up hyperref
\hypersetup{
  colorlinks,
  breaklinks,
  linkcolor=NavyBlue,
  urlcolor=NavyBlue,
  pdftitle={Thies Gehrmann PhD Summary},
  pdfauthor={Thies Gehrmann},
  pdfsubject={PhD Summary},
  pdfcreator={Thies Gehrmann},
  pdfproducer={Thies Gehrmann},
}

  % Geometry 
\textheight 23 cm
\textwidth 16 cm
\oddsidemargin 0 cm
\evensidemargin 0 cm

  % Create a function for quoting
\renewcommand{\quote}[1]{
  ``\emph{#1}''
}

  % Make lists without bullets
\renewenvironment{itemize}{
  \begin{list}{$\cdot$}{
    \setlength{\itemsep}{0pt}
    \setlength{\topsep}{7pt}
    \setlength{\leftmargin}{0pt}
    \setlength{\itemindent}{10pt}
  }
}{
  \end{list}
}

 % A new section header
\newcommand{\nsection}[1]{
    \section*{\color{MidnightBlue}#1}
  \vspace{-1mm}
}

  % A non italicized item in a section
\newcommand{\entryni}[2]{
  \noindent
  \begin{tabular}[t]{p{0.2\textwidth}|p{0.8\textwidth}}
    \textbf{\color{BrickRed}#1} & {#2} \\
  \end{tabular}
  \vspace{-2mm}

}

 % An italicized item in a section
\newcommand{\entry}[3]{
  \noindent
  \begin{tabular}[t]{p{0.2\textwidth}|p{0.8\textwidth}}
    \textbf{\color{BrickRed}#1} & {#2 \textit{#3}} \\
  \end{tabular}
  \vspace{-2mm}
}

\newcommand{\fl}[1]{
  {\color{MidnightBlue}\StrLeft{#1}{1}\xspace}\StrGobbleLeft{#1}{1}\xspace
}

  % No paragraph indent
\setlength\parindent{10pt}

  % Data definitions
\def\doctitle{\color{BrickRed}PhD Summary}
\def\name{\color{MidnightBlue}Thies Gehrmann}
\def\maxdeg{\color{Black}M.Sc}

%%%%%%%%%%%%%%%%%%%%%%%%%%%%%%%%%%%%%%%%%%%%%%%%%%%%%%%%%%%%%%%%%%%%%%%%%%%%%%%
%%                    END LATEX PREAMBLE                                     %%
%%%%%%%%%%%%%%%%%%%%%%%%%%%%%%%%%%%%%%%%%%%%%%%%%%%%%%%%%%%%%%%%%%%%%%%%%%%%%%%

\begin{document}

\small

\pagestyle{plain}

\label{top}

  % Title
{\huge {\textbf \name} {\footnotesize $\,$ \maxdeg} $\;$ {\textbf \doctitle} }
\vspace{0.1cm}
\hrule
\vspace{0.5cm}

%%%%%%%%%%%%%%%%%%%%%%%%%%%%%%%%%%%%%%%%%%%%%%%%%%%%%%%%%%%%%%%%%%%%%%%%%%%%%%%
%%                    CONTACT & CONTENT INFORMATION                          %%
%%%%%%%%%%%%%%%%%%%%%%%%%%%%%%%%%%%%%%%%%%%%%%%%%%%%%%%%%%%%%%%%%%%%%%%%%%%%%%%

\noindent\makebox[\textwidth][c]{%
\begin{minipage}[t]{0.9\textwidth}
  \setlength{\parindent}{2em}
  \setlength{\parskip}{1em}

  My PhD was embedded in the STW project 'Push the White Button', funded by the association of dutch mushroom growers.
  The goal of the project was to further the understanding of gene regulation in \textit{Scizophyllum commune} mushroom formation and to transfer that knowledge to the agricultural mushroom \textit{Agaricus bisporus}.

  In preparation of the transfer of knowledge, I developed Proteny, an algorithm to study the distant synteny between the two evolutionarily distant strains.
  The algorithm calculates a statistical significance for each discovered syntenic cluster, based on a background model of evolution.
  I found large chromosome similarities and rearrangements and duplications that would not have been found with traditional algorithms.

  Gene regulation was studied in these two organisms with RNA-Seq data.
  Therefore, I developed an RNA-Seq pipeline to analyze the gene expression from both strains.
  Through downstream analysis, we found a new transcription factor involved in mushroom fructification.
  We also verified that some fructification genes in \textit{S. commune} are also involved in fructification in \textit{A. bisporus}.

  Beyond the fundamental RNA-Seq analysis, I studied regulatory mechanisms that were quite unexplored in fungi.
  With the RNA-Seq data from \textit{S. commune}, we constructed alternatively spliced structures.
  Due to the gene density of this organism, traditional methods fuse genes together, resulting in incorrect structures.
  To overcome this, we developed a region-resistricted probablistic modeling method, whereby we restrict the search for structures to the regions of existing genes, preventing gene fusion.
  We find thousands of alternatively spliced transcripts which are all supported by sequencing data.
  By comparing the expression levels of the alternative transcripts of genes, we identify genes whose alternative splicing is regulated throughout development.
  With functional predictions, we also determine that alternative splicing produces alternative functionality in \textit{S. commune}.
  For the first time, we found that alternative splicing impacts all functional categories throughout mushroom development.

  In \textit{A. bisporus}, the nuclear organization of the genome is particularly interesting.
  Each cell contains on average six nuclei, each being  a copy of one of the two parental nuclei, referred to as the homokaryons of \textit{A. bisporus}.
  Genes therefore exist several times in two different forms, once in each homokaryon, and once in each nucleus.
  With our RNA-Seq data, and the recently sequenced homokaryons, we can study the homokaryon specific expression of these genes.
  I developed an algorithm to identify unique markers for each gene on the two homokaryons, and to quantify the abundance of these markers in each sample.
  We find hundreds of genes whose expression is significantly different between the two homokaryons.
  Additionally, the two different homokaryons are differentially active in mRNA production throughout mushroom tissue development.
  One homokaryon, known to have a white skin phenotype, is specifically active in skin tissue.
  Further, many chromosomes are also preferentially active in one homokaryon over the other.
  These facts indicate significant currently unknown regulatory mechanisms, and will impact mushroom breeding programmes.

  Throughout my PhD I focused on gene regulatory mechanisms that were previously ignored in mushrooms.
  I aided the discovery of additional transcription factors involved in mushroom development, and contributed to the transfer of knowledge from model mushroom to agricultural mushroom.
  Further, I found that two regulatory mechanisms, alternative splicing and homokaryon specific expression have substantial impacts throughout development.
  It is my desire to continue working to improve the wealth of knowledge on fungi, to expand their industrial applications in order to contribute positively to society.

\end{minipage}
}


\thispagestyle{empty}
\end{document}

