\documentclass[letterpaper, 10pt]{article} %norsk

% Packages
\usepackage[left=2cm, right=2cm, top=3cm]{geometry}
\usepackage{hyperref}
\usepackage{multicol}
\usepackage[usenames,dvipsnames]{color}
\usepackage{graphicx}
\usepackage{url}
\usepackage{longtable}
\usepackage{enumitem}
\usepackage{xspace}
\usepackage{xstring}
\usepackage[skins,breakable]{tcolorbox}

\usepackage{natbib}
\usepackage{bibentry}
\nobibliography{cv}

\usepackage{endnotes}

\let\footnote=\endnote

\makeatletter
\renewcommand\footnotesize{%
	\@setfontsize\footnotesize\@ixpt{3}%
	\abovedisplayskip 4\p@ \@plus2\p@ \@minus4\p@
	\abovedisplayshortskip \z@ \@plus\p@
	\belowdisplayshortskip 4\p@ \@plus2\p@ \@minus2\p@
	\def\@listi{\leftmargin\leftmargini
		\topsep 4\p@ \@plus2\p@ \@minus2\p@
		\parsep 2\p@ \@plus\p@ \@minus\p@
		\itemsep \parsep}%
	\belowdisplayskip \abovedisplayskip
}
\makeatother

% Set up hyperref
\hypersetup{
	colorlinks,
	breaklinks,
	linkcolor=NavyBlue,
	urlcolor=NavyBlue,
	pdftitle={Thies Gehrmann Letter of Motivation},
	pdfauthor={Thies Gehrmann},
	pdfsubject={Letter of Motivation},
	pdfcreator={Thies Gehrmann},
	pdfproducer={Thies Gehrmann},
}

% Geometry 
\textheight 23 cm
\textwidth 16 cm
\oddsidemargin 0 cm
\evensidemargin 0 cm

% Create a function for quoting
\renewcommand{\quote}[1]{
	``\emph{#1}''
}

% Make lists without bullets
\renewenvironment{itemize}{
	\begin{list}{$\cdot$}{
			\setlength{\itemsep}{0pt}
			\setlength{\topsep}{7pt}
			\setlength{\leftmargin}{0pt}
			\setlength{\itemindent}{10pt}
		}
	}{
	\end{list}
}

% A new section header
\newcommand{\nsection}[1]{
	\section*{\color{MidnightBlue}#1}
	\vspace{-1mm}
}

% A non italicized item in a section
\newcommand{\entryni}[2]{
	\noindent
	\begin{tabular}[t]{p{0.2\textwidth}|p{0.8\textwidth}}
		\textbf{\color{BrickRed}#1} & {#2} \\
	\end{tabular}
	\vspace{-2mm}
	
}

% An italicized item in a section
\newcommand{\entry}[3]{
	\noindent
	\begin{tabular}[t]{p{0.2\textwidth}|p{0.8\textwidth}}
		\textbf{\color{BrickRed}#1} & {#2 \textit{#3}} \\
	\end{tabular}
	\vspace{-2mm}
}

\newcommand{\fl}[1]{
	{\color{MidnightBlue}\StrLeft{#1}{1}\xspace}\StrGobbleLeft{#1}{1}\xspace
}

% No paragraph indent
\setlength\parindent{10pt}

% Data definitions
\def\doctitle{\color{BrickRed}Letter of Motivation}
\def\name{\color{MidnightBlue}Thies Gehrmann}
\def\maxdeg{\color{Black}Ph.D}


%%%%%%%%%%%%%%%%%%%%%%%%%%%%%%%%%%%%%%%%%%%%%%%%%%%%%%%%%%%%%%%%%%%%%%%%%%%%%%%
%%                    END LATEX PREAMBLE                                     %%
%%%%%%%%%%%%%%%%%%%%%%%%%%%%%%%%%%%%%%%%%%%%%%%%%%%%%%%%%%%%%%%%%%%%%%%%%%%%%%%

\begin{document}

\small

\pagestyle{plain}

\label{top}

  % Title
{\huge {\textbf \name} {\footnotesize $\,$ \maxdeg} $\;$ {\textbf \doctitle} }
\vspace{0.1cm}
\hrule
\vspace{0.5cm}

%%%%%%%%%%%%%%%%%%%%%%%%%%%%%%%%%%%%%%%%%%%%%%%%%%%%%%%%%%%%%%%%%%%%%%%%%%%%%%%
%%                    CONTACT & CONTENT INFORMATION                          %%
%%%%%%%%%%%%%%%%%%%%%%%%%%%%%%%%%%%%%%%%%%%%%%%%%%%%%%%%%%%%%%%%%%%%%%%%%%%%%%%


\begin{tcolorbox}[
	blanker,
	width=0.95\textwidth,
	enlarge left by=0.025\textwidth,
	enlarge right by=0.025\textwidth,
	before skip=6pt,
	breakable]
\setlength{\parindent}{2em}
\setlength{\parskip}{1em}


\noindent Dear Prof. Raes, and to whom else it may concern,

I am applying to your open position for a postdoctoral bioinformatian. Currently I am a postdoc in the Molecular Epidemiology group at the Leiden University Medical Center in The Netherlands, and the Max Planck Institute for the Biology of Ageing in Cologne, Germany. Here, I have worked on identifying longevity-associated genomic variants in families of long-lived individuals, which we are currently following up in model systems. I also examine the relationship between transcriptomic and metabolomic effects of nutrient intake and how it is affected by a combined dietary and physical lifestyle intervention in different tissues of healthy elderly male and female humans. Previously, I obtained my PhD from the Delft University of Technology where I studied nuclear-specific expression, alternative splicing, synteny and mutation accumulation, and how these factors are related to mushroom fruiting body development, and the study thereof. Clearly, I have a broad interest as far as the underlying biology is concerned, and I am excited to apply to your position, which provides an opportunity to combine my microbiological past, and my human biology present by way of the metagenomic axis.

I have not yet had the opportunity to develop my research in this direction, but it has always been present in my research periphery, and has been a long standing interest of mine - I believe one of the first exposures I had to it was the famous Tara Oceans project in which you also participated. Of particular interest to me was the absence of fungi from almost all metagenomic studies, limited perhaps to a few yeasts. This, despite fungi being a major source of nutrients in key ecologies - where there is lignin, there are fungi, and entire ecologies can flourish around them. During a postdoc at the Dutch Fungal Biodiversity Center, I learned why they were so ignored: Fungal cell structures are very variable, and methods to reliably isolate DNA from fungi and bacteria invariably leave many fungal cells intact, and bacterial DNA shredded beyond use. This was evident when, in a collaboration with the Leiden University Medical Center, we found hardly any fungal reads in their whole-metagenome sequencing data of the human gut. I also participated in the construction of the ITS/LSU barcode database for the fungal collection of the biodiversity institute, which we hoped would aid future metagenomic endeavours in the fungal kingdom. When I commenced my research in longevity, the relationship between gut bacterial communities and age was immediately clear. Biological clocks that predict age have been constructed from methylation, metabolomics and gut microbiomes. My colleagues at the Max Planck Institute for the Biology of Ageing have demonstrated that fecal transplants from young killifish extend the lifespan of middle aged fish. Despite not having much hands on experience with metagenomics, my proximity to this field has left me with knowledge of the methodologies generally applied in their analysis. It has also instilled in me a deep appreciation for the need for further research into the relationships within and between communities and their ecologies that surround us and sustain life on earth.

In the job announcement, there is no specific project described, from which I deduce that the position is open-ended. From the datasets specified in the advertisement, and the papers coming from your lab, I envisage many possibilities. For example, in your paper on GMMs, you stopped short of associating GMMS, or measures derived from them such as Functional redundancy, to health indicators (e.g. BMI, blood pressure) in the MetaHIT study - To what extent do such functional capacity indicators capture different aspects of health? Within the context of Tara Ocean's Mission Microplastics, how do the types and prevalence of microplastics, affect the plankton communities and interactomes? From my own experience in metabolomic and especially transcriptomic studies, sex specific effects are not suitably addressed in linear models. Instead, I have found it necessary to investigate effects seperately in males and females. With this in mind, how do the metabolomes and transcriptomes relate to each other differently in males and females, for example in the Lifelines Deep study? As far as I am aware, these are open questions that may fit within the research line of your group.

You advertise your lab as a \textbf{\textit{"stimulating and fun environment to pursue your scientific dreams"}}. This is an inspiring promise, and in the spirit of your statement would like to share some of my scientific dreams with you. Within the field of metagenomics, there are three projects which I have long dreamed of. They are longer-term projects, for which I would like to pursue funding. That your group is situated in the microbiology department is something that I would take advantage of for the collaborations necessary in these projects.

1. The first considers a shift in how we investigate the relationship between microbiota and their hosts. Typically, a dysbiosis is identified, and associated with whatever status, marker or omics data is of interest. This largely ignores the metabolomic necessity of the microbiota, and does not help us achieve a mechanistic understanding of the relationship between the ecologies and their host environments. There have been some steps towards a mechanistic understanding, among them the gut metabolic modules (GMMs) that your group has developed, and also the Genome-Scale Metabolic Models developed (GSMMs) in the group of Bas Dutilh. From your manuscript on the \textit{Synthetic ecology of the human gut microbiota}, I think that you agree that work in this direction should continue. I would like to explore these concepts further, and develop tools that enable us to build, from the predicted or experimentally determined metabolic capacities of a given microbiome, up towards the functional statuses of their hosts. To achieve this, I propose to describe the functional capacity of each species in terms of functional modules, similar to the GMM concept, but ideally on a continuous scale, perhaps through the use of flux or kinetic models. With this, we could make an effort to explain which bacteria contribute how much functional capacity to the biome, and how it relates to systemic features of the host? I would very much like to develop this admittedly crude idea further.

2. The second relates to the effects of different agricultural practices on soil quality. As long as humans have been an agrarian species, we have relied on the soils in which our drops grow. The optimization of agricultural processes towards low cost, high yield farming practices have contributed massively towards land degradation. A transition to alternative agricultural techniques is necessary to sustain human life on earth. Intercropping is an agricultural technique that exploits symbiotic relationships between plants, either in phenotype or in chemistry, to improve the yield of a plot of land, either per harvest or over many cycles. These symbiotic relationships have been neglected in modern agriculture, but have been developed through centuries of traditional agriculture, and are revived in alternative agriculture (poly- and perma-culture). The impact of growing several species on the same plot on soil has not been properly investigated in a metagenomic capacity before. In a polyculture plot, due to the different demands from the soil, I anticipate a selection for different soil biodiversities that result in different levels of soil health when compared to monoculture plots. An investigation of this would yield insights into plant-microbe-plant interactions, and inform strategies for an agricultural shift to prevent the effects of unsustainable agricultural practices.

3. In the third, I propose that the exploitation of bacteria is a part of our human heritage. How have we (indirectly) selected our grains based on the microbial communities that grow on them? Some bacteria may protect the grains from fungi as they dry, others are vital in preservation and processing of flour into bread. We often discuss the selected breeding of crops as something that we do on the crop itself, whereas we do it on the entire ecology surrounding it - the bacteria and fungi on and in the grains and in the soil. Today, in the age of extreme pesticide use, we may be in danger of losing the thousands of years of selection our ancestors have made to arrive at a grain whose microbial ecology benefited us (and our soils). How does the microbial ecology of ancient grains look like, and how did they co-evolve? I anticipate that a co-operation with seed banks could help unravel our agro-microbial heritage.

I hope that this letter has conveyed some of my interest, enthusiasm, and suitability for research in your lab. I look forward to hearing from you, and I hope that I may have the opportunity to discuss the ideas in this letter further with you.

Kind regards,

-Thies Gehrmann
\end{tcolorbox}



\thispagestyle{empty}
\end{document}

